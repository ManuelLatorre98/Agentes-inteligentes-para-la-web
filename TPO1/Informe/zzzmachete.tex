\section{EO}
\begin{lstlisting}[style=java,caption= Metodo ordenarPorConteoChar]
  public class M { 
  public int a, b;
  static int c = 9; 
 \end{lstlisting}

 \begin{figure}[!htb]
  \centering
  \includegraphics[width=16cm, scale=1]{Images/Punto2/Marcado.png}
  \caption{Estado de memoria y desarrollo de proceso de marcado}
  \label{fig:marcado}
\end{figure}





 \begin{lstlisting}[
  language=SQL,
  showspaces=false,
  basicstyle=\ttfamily,
  numbers=left,
  numberstyle=\tiny,
  commentstyle=\color{gray}
]
a = LOAD 'data' USING BinStorage AS (user);
b = GROUP a BY user;
/* Now we are ready to loop */
c = FOREACH b GENERATE COUNT(a) AS cnt;
d = ORDER c BY cnt;
\end{lstlisting}


\begin{tcolorbox}
  \begin{center}
   \textbf{Dado un rompecabezas Battleships $\langle I, C, R, F\rangle$ \\ ¿Tiene este rompecabezas una solución?}
  \end{center}
\end{tcolorbox}


//Tablas
\begin{figure}[htb]
  \centering
  \scriptsize %%tamaño fuente
  \small
  \begin{tabular}{| p{4.5 cm} | p{12 cm} |}
  \hline
  \textbf{Nombre} & asdasd \\
  \hline
  \textbf{Actores}  & asdasd \\
  \hline
  \textbf{Precondiciones} & asdasd \\
  \hline
  \textbf{Postcondiciones} & asdasda \\
  \hline
  \textbf{Escenario básico} & Fila 2, Columna 2 \\
  \hline
  \textbf{Escenarios alternativos} & Fila 2, Columna 2 \\
  \hline
  \textbf{escenarios de excepción} & Fila 2, Columna 2 \\
  \hline
  \end{tabular}
\end{figure}